\NeedsTeXFormat{LaTeX2e} [1994/06/01]
\documentclass{article}
%%\usepackage [pagewise, mathlines]{lineno}
\usepackage{graphicx}
\usepackage[reqno,intlimits,sumlimits]{amsmath}

\renewcommand{\baselinestretch}{1.2}
\renewcommand{\arraystretch}{1}

\setlength{\oddsidemargin}{0.0in} \setlength{\hoffset}{0in}
\setlength{\textwidth}{6.5in}
\setlength{\topmargin}{0cm} \setlength{\voffset}{0.0in}
\setlength{\textheight}{9in}

%%% Needed to get lineno to work nicely with amsmath
%%\newcommand*\patchAmsMathEnvironmentForLineno[1]{
%% \expandafter\let\csname old#1\expandafter\endcsname\csname #1\endcsname
%% \expandafter\let\csname oldend#1\expandafter\endcsname\csname end#1\endcsname
%% \renewenvironment{#1}
%% {\linenomath\csname old#1\endcsname}
%% {\csname oldend#1\endcsname\endlinenomath}}
%%\newcommand*\patchBothAmsMathEnvironmentsForLineno[1]{
%% \patchAmsMathEnvironmentForLineno{#1}
%% \patchAmsMathEnvironmentForLineno{#1*}}
%%\AtBeginDocument{
%%\patchBothAmsMathEnvironmentsForLineno{equation}
%%\patchBothAmsMathEnvironmentsForLineno{align}}

\title{An alternate formulation of the unsaturated zone solution given in Mishra-Neuman (2010)}
\begin{document}
\maketitle
\section{Unsaturated Zone Governing Equations}
The governing equation in the unsaturated zone, as used by \cite{mishra10} and \cite{tartakovsky07}, is
\begin{equation}
  \label{eq:unsatDim}
  K_r k_o(z) \frac{1}{r} \frac{\partial}{\partial r} \left( r\frac{\partial \sigma}{\partial r} \right) + K_z \frac{\partial}{\partial z} \left( k_0(z) \frac{\partial \sigma}{\partial z}\right) = C_0(z) \frac{\partial \sigma}{\partial t}
\end{equation}
over the unsaturated zone, defined as $b \le z \le b+L$.  Further, the unsaturated parameters are assumed to only be functions of $z$, $k_0(z)=k_0(\theta_0)$ and $C_0(z)=C_0(\theta_0)$.  The initial and boundary conditions for the unsaturated zone are  
$$\sigma(r,z,0) = 0$$
$$\sigma(\infty,z,t)=0$$
$$\frac{\partial \sigma}{\partial z}=0 \qquad z=b+L$$ and 
$$\lim_{r \rightarrow 0} r \frac{\partial \sigma}{\partial r} = 0 \qquad b\le z \le b+L$$

Drawdown in the vadose zone is defined as $\sigma = h_0 - h=b+\psi_a - h$, $0 \le k_0 \le 1$ is the isotropic relative hydraulic conductivity, $C(\psi)=\frac{d\eta}{d \psi}$ is the specific moisture capacity and $\theta$ is volumetric water content.  Drawdown in the aquifer, $s$, and the vadose zone are connected using compatibility conditions at the water table
$$s=\sigma \qquad z=b$$
$$\frac{\partial s}{\partial z}=\frac{\partial \sigma}{\partial z} \qquad z=b$$ 

The moisture retention curve in the vadose zone is represented as an exponential function
$$S_e = \frac{\theta(\psi) - \theta_r}{S_y} = e^{a_c \left( \psi - \psi_a \right)} \qquad a_c \ge 0 $$
where $S_e$ is effective saturation, $\theta_r$ is residual water content, and $S_y=\theta_s - \theta_r$ is drainable porosity (specific yield).  The Gardner model for relative hydraulic conductivity is used,
\begin{equation}
  \nonumber
  k(\psi) = \left  \{ 
    \begin{array}{c c}
      e^{a_k \left( \psi - \psi_k\right)} & \psi \le \psi_k \\
      1 & \psi < \psi_k\\
    \end{array} 
\right . \qquad a_k \ge 0.
\end{equation}
The 4-parameter exponential model for hydraulic conductivity used in the vadose zone is 
\begin{equation}
  \label{eq:Gardner}
 k_0(z)=e^{a_k\left( b_1 + b - z\right)} \qquad b_1=\psi_a-\psi_k
\end{equation}
and the moisture capacity is
\begin{equation}
  \label{eq:mrc}
C_0(z) = S_y a_c e^{a_c \left( b-z\right)}
\end{equation}
\section{Unsaturated Zone Solution}
The Laplace transformation (overbar and $p$) of (\ref{eq:unsatDim}) results in
\begin{equation}
  \label{eq:unsatLap}
   k_o(z) \frac{1}{r} \frac{\partial}{\partial r} \left( r\frac{\partial \bar{\sigma}}{\partial r} \right) +\kappa \frac{\partial}{\partial z} \left( k_0(z) \frac{\partial \bar{\sigma}}{\partial z}\right) = p \frac{C_0(z)}{K_r}  \bar{\sigma}
\end{equation}
where $\kappa=K_z/K_r$, with the associated transformed boundary and initial conditions of
$$ \bar{\sigma}(\infty,z,p) = 0$$
$$ \frac{\partial \bar{\sigma}}{\partial z}=0 \qquad z=b+L$$
 $$\lim_{r \rightarrow 0} r \frac{\partial \bar{\sigma}}{\partial r} = 0 \qquad b\le z \le b+L$$
The Hankel transformation (superscript $\ast$ and $a$) of (\ref{eq:unsatLap}) and substitution of the constituative models (\ref{eq:Gardner}) and (\ref{eq:mrc}) results in the ordinary differential equation
\begin{equation}
  \label{eq:unsatHank}
   -a^2 \bar{\sigma}^{\ast} + \kappa a_k \frac{\mathrm{d}^2 \bar{\sigma}^{\ast}}{\mathrm{d} z^2}  = p e^{-a_k b_1} \frac{S_y a_c} {K_r}  e^{\left(a_k -a_c \right)\left( b - z\right)} \bar{\sigma}^{\ast} 
\end{equation}
and the no-flow condition at the top of the vadose zone, 
$$ \frac{\mathrm{d}\bar{\sigma}^{\ast}}{\mathrm{d}z}=0 \qquad z=b+L $$
Equation \ref{eq:unsatHank} can be rearranged as
\begin{equation}
  \label{eq:mn-d5}
  \frac{\mathrm{d}^2 \bar{\sigma}^{\ast}}{\mathrm{d}z^2} - a_k \frac{\mathrm{d} \bar{\sigma}^{\ast}}{\mathrm{d}z} - \left( B e^{\lambda (z-b)} + C\right) \bar{\sigma}^{\ast}=0
\end{equation}
where $\lambda = a_k-a_c$, $C=\frac{a^2}{\kappa}$ and
$$B = p e^{-a_k b_1} \frac{S_y a_c} {K_r} $$
which was solved in \cite{mishra10} using a general solution from a reference book on differential equations.  The solution given in terms of two types of Bessel functions of complex argument and non-integer order is non-trivial to evaluate numerically. 
\subsection{Modified Solution Procedure} 
By non-dimensionalizing the $z$-coordinate and performing an exponential substitution, (\ref{eq:mn-d5}) can be simplified significantly and a solution can be found by integration.  Using the new dimensionless vertical coordinate 
$$ \frac{\zeta}{z}=a_k $$
the transformed differential equation representing drawdown in the vadose zone (\ref{eq:mn-d5}) becomes
\begin{equation}
  \label{eq:nondimODE}
  a_k^2 \frac{\mathrm{d}^2 \bar{\sigma}^{\ast}}{\mathrm{d}\zeta^2} - a_k^2 \frac{\mathrm{d} \bar{\sigma}^{\ast}}{\mathrm{d}\zeta} - \left( B e^{\lambda (\zeta/a_k-b)} + C\right) \bar{\sigma}^{\ast}=0
\end{equation}
dividing through by $a_k^2$ and performing the substitution $\bar{\sigma}^{\ast}(z)=e^{u(z)}$ (which leads to $\frac{\mathrm{d}\bar{\sigma}^{\ast}}{\mathrm{d}\zeta} = \frac{\mathrm{d}u}{\mathrm{d}\zeta}e^u$ and $\frac{\mathrm{d}^2\bar{\sigma}^{\ast}}{\mathrm{d}\zeta^2} = \frac{\mathrm{d}^2u}{\mathrm{d}\zeta^2}e^u + \frac{\mathrm{d}u}{\mathrm{d}\zeta}e^u$) gives
\begin{equation}
  \label{eq:expsubODE}
   \frac{\mathrm{d}^2 u}{\mathrm{d}\zeta^2} = \left( \frac{B}{a_k^2} e^{\lambda (\zeta/a_k-b)} + \frac{C}{a_k^2}\right) 
\end{equation}

This is an integrable equation; one indefinite integraion produces
$$ \frac{\mathrm{d} u}{\mathrm{d}\zeta} =  \frac{B}{\lambda a_k} e^{\lambda (\zeta/a_k-b)} + \frac{C}{a_k^2}\zeta + \alpha  $$
where $\alpha$ is a constant of integration. A second integration produces
$$ u =  \frac{B}{\lambda^2} e^{\lambda (\zeta/a_k-b)} + \frac{C}{2a_k^2}\zeta^2 + \alpha\zeta + \beta  $$
where $\beta$ is another constant of integration.  

Applying the exponential transformation to the boundary condition at the top of the vadose zone results in
$$\frac{\mathrm{d}u}{\mathrm{d}\zeta}=0 \qquad \zeta=ba_k + La_k$$
which is used to determine one of the constants
$$-\alpha= \frac{B}{\lambda a_k} e^{\lambda L} + \frac{C}{a_k} \left( b+L\right) $$

The solution for the drawdown in the vadose zone is then
\begin{equation}
  \nonumber
  \bar{\sigma}^{\ast} (z) = \tilde{\beta}\exp\left\{ \frac{B}{\lambda^2} e^{\lambda (z-b)} + \frac{C}{2}z^2 -\left[\frac{B}{\lambda} e^{\lambda L} + C \left( b+L\right)\right]z  \right\}
\end{equation}
where $\tilde{\beta}=e^\beta$ is a different integration constant. This can be re-grouped and simplified as
\begin{equation}
  \label{eq:sigma}
  \bar{\sigma}^{\ast} (z) = \tilde{\beta}\exp\left\{  \underbrace{\frac{p e^{-a_k b_1} S_y a_c} {K_r \lambda^2} \left[ e^{\lambda (z-b)}  - \lambda e^{\lambda L} \right]}_{\rm Laplace} + \underbrace{\frac{a^2 z}{\kappa}\left[\frac{z}{2} - \left( b+L\right) \right]}_{\rm Hankel}  \right\}
\end{equation}
and $\tilde{\beta}$ will be determined from the matching conditions at the water table.
\section{Saturated Zone Unconfined Solution}
Follosing \cite{mishra10}, we decompose the saturated zone solution into two solutions: 1) the Hantush solution $s_H$ for a confined aquifer and a partially penetrating well; 2) an unconfined solution $s_U$ with no well but continuity conditions with the vadose zone solution just derived.  The solution for $s_H$ is given elsewhere, either in terms of a finite cosine transform, or a multi-layer fully penetrating solution.  

The governing equation  in Laplace-Hankel transform space for $s_U$ is
\begin{equation}
  \label{eq:aquifer}
  \frac{\partial^2 \bar{s}_U^{\ast}}{\partial z^2} - \left( \frac{p}{\alpha_s \kappa} + \frac{a^2}{\kappa} \right)\bar{s}_U^{\ast} = 0 \qquad 0 \le z \le b
\end{equation}
with the no-flow boundary condition at the base of the aqufier
$$ \frac{\partial \bar{s}_U^{\ast}}{\partial z} = 0 \qquad z=0 $$
and the following continuity equations at the water table 
$$ \bar{s}_U^{\ast} + \bar{s}_H^{\ast} = \bar{\sigma}^{\ast} \qquad z=b$$
$$ \frac{\partial \bar{s}_U^{\ast}}{\partial z}  = \frac{\partial \bar{\sigma}^{\ast}}{\partial z} \qquad z=b $$
where the vertical flux of $\bar{s}_H^{\ast}$ is defined as zero at the top and bottom of the aqufier, so it is not included in the continuity conditions.

The general solution to \eqref{eq:aquifer} is 
\begin{equation}
  \label{eq:su}
  \bar{s}_U^{\ast} = A_1 \cosh(\eta z) + A_2 \sinh(\eta z)
\end{equation}
where $A_i$ are constants to determine and $\eta^2 = \frac{p}{\alpha_s \kappa} + \frac{a^2}{\kappa}$.  The no-flow boundary condition at the bottom of the aquifer forces $A_2=0$.  

\section{New Solution using New Results for Unsaturated Zone}
Substituting \eqref{eq:sigma} and \eqref{eq:su} into the compatibility equations gives
$$  A_1 \cosh(\eta b) + \bar{s}_H^{\ast}(b) = \tilde{\beta}e^{\omega_1}$$
$$ A_1 \eta  \sinh(\eta b)  = \tilde{\beta} \omega_2 e^{\omega_1}$$
where 
$$ \omega_1 = \frac{p e^{-a_k b_1} S_y a_c} {K_r \lambda^2} \left[ 1  - \lambda e^{\lambda L} \right] + \frac{a^2 b}{\kappa}\left[\frac{b}{2} - \left( b+L\right) \right]$$
\begin{equation}
  \label{eq:omega2}
  \omega_2 = \frac{p e^{-a_k b_1} S_y a_c} {K_r \lambda} \left[ 1  -  e^{\lambda L} \right] + \frac{a^2 b}{\kappa}\left[1 - \left( b+L\right) \right]
\end{equation}
Divide the second equation by $\omega_2$ and set the left-hand-sides of the two resulting equations equal to one another.  Solving for $A_1$ leads to
$$A_1 \cosh(\eta b) + \bar{s}_H^{\ast}(b) = A_1 \frac{\eta}{\omega_2}  \sinh(\eta b) $$
$$ A_1 = \frac{\bar{s}_H^{\ast}(b)}{\frac{\eta}{\omega_2}  \sinh(\eta b) - \cosh(\eta b)}$$
which gives the final Laplace-Hankel form of the saturated-zone solution as
$$ \bar{s}^{\ast}(z)  = \bar{s}_H^{\ast}(z) + \bar{s}_H^{\ast}(b) \frac{\cosh(\eta z)}{\frac{\eta}{\omega_2}  \sinh(\eta b) - \cosh(\eta b)}$$ 
This expression is identical in form to \cite[eqn.\ C17]{mishra10}, with $\omega_2$ instead of $q$.  The expression for $q$ \cite[eqn.\ D13]{mishra10} involves ratios of Bessel functions of complex argument and non-integer order, while \eqref{eq:omega2} only involves exponentials.

The final Laplace-Hankel form of the unsaturated zone solution can be found by substituting $A_1$ into one of the expressions for $\tilde\beta$ above, and substituting that into \eqref{eq:sigma}.
\bibliographystyle{alpha}
\bibliography{mn}

\end{document}
