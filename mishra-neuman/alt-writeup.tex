\NeedsTeXFormat{LaTeX2e} [1994/06/01]
\documentclass[12pt,letterpaper]{article}
\usepackage[mathlines,pagewise]{lineno}
\usepackage{graphicx}
\usepackage{amsmath}

\renewcommand{\baselinestretch}{1.2}
\renewcommand{\arraystretch}{1}

\setlength{\oddsidemargin}{0.0in} \setlength{\hoffset}{0in}
\setlength{\textwidth}{6.5in}
\setlength{\topmargin}{0cm} \setlength{\voffset}{0.0in}
\setlength{\textheight}{8.25in}

% Needed to get lineno to work nicely with amsmath
\newcommand*\patchAmsMathEnvironmentForLineno[1]{
 \expandafter\let\csname old#1\expandafter\endcsname\csname #1\endcsname
 \expandafter\let\csname oldend#1\expandafter\endcsname\csname end#1\endcsname
 \renewenvironment{#1}
 {\linenomath\csname old#1\endcsname}
 {\csname oldend#1\endcsname\endlinenomath}}
\newcommand*\patchBothAmsMathEnvironmentsForLineno[1]{
 \patchAmsMathEnvironmentForLineno{#1}
 \patchAmsMathEnvironmentForLineno{#1*}}
\AtBeginDocument{
\patchBothAmsMathEnvironmentsForLineno{equation}
\patchBothAmsMathEnvironmentsForLineno{align}}

\title{Alternate Formulations for Mishra-Neuman Solution}
\begin{document}
\maketitle
\linenumbers
\section{Unsaturated Zone Governing Equations}
The governing equation in the unsaturated zone, as used by \cite{mishra10} and \cite{tartakovsky07}, is
\begin{equation}
  \label{eq:unsatDim}
  K_r k_0(\psi) \frac{1}{r} \frac{\partial}{\partial r} \left( r\frac{\partial \sigma}{\partial r} \right) + K_z \frac{\partial}{\partial z} \left( k_0(z) \frac{\partial \sigma}{\partial z}\right) = C(\psi) \frac{\partial \sigma}{\partial t}
\end{equation}
where $K_r$ and $K_z$ are the radial and vertical saturated hydraulic conductivities [L T$^{-1}$], $k_0$ is the dimensionless isotropic relative hydraulic conductivity ($0 < k_0 \le 1$), $\sigma = h_0 - h = b +\psi_a - h$ is drawdown in the vadose zone [L], $h=\psi+z$ is hydraulic head [L], $\psi$ is pressure head [L], $\psi_a$ is air-entry pressure, $C(\psi)=\frac{d\theta}{d \psi}$ is the dimensionless specific moisture capacity, and $\theta$ is dimensionless volumetric water content.  The unsaturated zone begins at the top of the aquifer, $b \le z \le b+L$.  Further, the unsaturated parameters are assumed to only be functions of $z$.  The initial and boundary conditions for the unsaturated zone are  
\begin{equation}\nonumber
\sigma(r,z,0) = 0
\end{equation}
\begin{equation}\nonumber
\sigma(\infty,z,t)=0
\end{equation}
\begin{equation}\nonumber
\frac{\partial \sigma}{\partial z}=0 \qquad z=b+L
\end{equation} and 
\begin{equation}\nonumber
\lim_{r \rightarrow 0} r \frac{\partial \sigma}{\partial r} = 0 \qquad b\le z \le b+L.
\end{equation}

Drawdown in the aquifer ($s$) and the vadose zone are connected using compatibility conditions at the water table ($z=b$), namely
\begin{equation}\nonumber
s=\sigma \qquad z=b
\end{equation}
\begin{equation}\nonumber
\frac{\partial s}{\partial z}=\frac{\partial \sigma}{\partial z} \qquad z=b.
\end{equation} 

The moisture retention curve in the vadose zone is modeled as an exponential function
\begin{equation}\nonumber
S_e = \frac{\theta(\psi) - \theta_r}{S_y} = e^{a_c \left( \psi - \psi_a \right)} \qquad a_c \ge 0 
\end{equation}
where $S_e$ is dimensionless effective saturation, $\theta_r$ is dimensionless residual volumetric water content, $S_y=\theta_s - \theta_r$ is dimensionless drainable porosity (specific yield), and $\theta_s$ is dimensionless saturated volumetric water content.  The Gardner model for relative hydraulic conductivity is used; the relative hydraulic conductivity is
\begin{equation}
  \nonumber
  k(\psi) = \left  \{ 
    \begin{array}{c c}
      e^{a_k \left( \psi - \psi_k\right)} & \psi \le \psi_k, \ \\
      1 & \psi < \psi_k\\
    \end{array} 
\right . \qquad a_k \ge 0,\; \psi_k \le 0,
\end{equation}
where $\psi_k$ is the pressure head beyond which the hydraulic conductivity is essentially saturated.  

The four-parameter exponential model for hydraulic conductivity used in the vadose zone is 
\begin{equation}
  \label{eq:Gardner}
 k_0(z)=e^{a_k\left( \Psi + b - z\right)} \qquad \Psi=\psi_a-\psi_k
\end{equation}
and the exponential model use for moisture capacity is
\begin{equation}
  \label{eq:mrc}
C_0(z) = S_y a_c e^{a_c \left( b-z\right)}.
\end{equation}

The governing equation will be non-dimensionalized before proceeding further with the solution. Equation \eqref{eq:unsatDim} can be written in dimensionless form as
\begin{equation}
  \label{eq:unsatDimless}
  k_0(z_D) \frac{1}{r_D} \frac{\partial}{\partial r_D} \left( r_D\frac{\partial \sigma_D}{\partial r_D} \right) + \kappa \frac{\partial}{\partial z_D} \left( k_0(z_D) \frac{\partial \sigma_D}{\partial z_D}\right) = \gamma C_0(z_D) \frac{\partial \sigma_D}{\partial t_D}
\end{equation}
where $r_D=r/b$ is the dimensionless radial coordinate, $z_D=z/b$ is the dimensionless vertical coordinate, $\kappa=K_z/K_r$ is the anisotropy ratio, $\sigma_D = \sigma/H_c$ is dimensionless vadose zone drawdown, $H_C$ is a characteristic head, $t_D = t K_r / (b^2 S_S)$ is dimensionless time, $\gamma = S_y a_c/S_S$ is the dimensionless storage ratio, and $S_S$ is the specific storage [L$^-1$] in the saturated aquifer.  The hydraulic conductivity constitutive model is non-dimensionalized as
\begin{equation}
  \label{eq:GardnerDimless}
 k_0(z_D)=e^{a_{kD} \left( \Psi_D + 1 - z_D \right)},
\end{equation}
where $a_{kD} = a_k b$ and $\Psi_D=\Psi/b$, and the moisture capacity model is non-dimensionalized as
\begin{equation}
  \label{eq:mrcDimless}
C_0(z_D) = e^{a_{cD} \left( 1-z_D\right)},
\end{equation}
where $a_{cD} = a_c b$.

\section{Unsaturated Zone Solution}
The dimensionless Laplace transformation (over-bar and $p$) of (\ref{eq:unsatDimless}) results in
\begin{equation}
  \label{eq:unsatLap}
   k_0(z_D) \frac{1}{r_D} \frac{\partial}{\partial r_D} \left( r_D\frac{\partial \bar{\sigma}_D}{\partial r_D} \right) +\kappa \frac{\partial}{\partial z_D} \left( k_0(z_D) \frac{\partial \bar{\sigma}_D}{\partial z_D}\right) = p \gamma C_0(z_D)  \bar{\sigma}_D.
\end{equation}
The associated transformed dimensionless boundary and initial conditions of
\begin{equation}\nonumber
 \bar{\sigma}_D(\infty,z_D,p) = 0
\end{equation}
\begin{equation}\nonumber
 \frac{\partial \bar{\sigma}_D}{\partial z_D}=0 \qquad z_D=1+L_D
\end{equation}
 \begin{equation}\nonumber
\lim_{r_D \rightarrow 0} r_D \frac{\partial \bar{\sigma}_D}{\partial r_D} = 0 \qquad 1\le z_D \le 1+L_D,
\end{equation}
where $L_D = L/b$ is the dimensionless vadose zone thickness.

Further applying the dimensionless Hankel transformation (superscript $\ast$ and $a$) to (\ref{eq:unsatLap}), followed by substitution of the dimensionless constitutive models (\ref{eq:GardnerDimless}) and (\ref{eq:mrcDimless}) results in the ordinary differential equation
\begin{equation}
  \label{eq:unsatHank}
   -a^2 \bar{\sigma}^{\ast} - \kappa a_{kD} \frac{\mathrm{d} \bar{\sigma}_D^{\ast}}{\mathrm{d} z_D} + \kappa \frac{\mathrm{d}^2 \bar{\sigma}_D^{\ast}}{\mathrm{d} z_D^2}  = p \gamma \bar{\sigma}_D^{\ast}  e^{-a_{kD} \Psi_D}  e^{\left(a_{kD} -a_{cD} \right)\left( 1 - z_D\right)}  
\end{equation}
and the no-flow condition at the top of the vadose zone, 
\begin{equation}\nonumber
 \frac{\mathrm{d}\bar{\sigma}_D^{\ast}}{\mathrm{d}z_D}=0 \qquad z_D=1+L_D 
\end{equation}
Equation \ref{eq:unsatHank} can be rearranged and regrouped as
\begin{equation}
  \label{eq:mn-d5}
  \frac{\mathrm{d}^2 \bar{\sigma}_D^{\ast}}{\mathrm{d}z_D^2} - a_{kD} \frac{\mathrm{d} \bar{\sigma}_D^{\ast}}{\mathrm{d}z_D} - \left[ B e^{\lambda_D (1-z_D)} + C\right] \bar{\sigma}^{\ast}=0
\end{equation}
where $\lambda_D = a_{kD} - a_{cD}$, $B = \frac{p\gamma}{\kappa} e^{-a_{kD} \Psi_D}$, and $C=\frac{a^2}{\kappa}$.

The dimensional form of \eqref{eq:mn-d5} was solved in \cite{mishra10} using a general solution from a
reference book on differential equations.  The solution given in terms
of two types of Bessel functions of complex argument and non-integer
order is non-trivial to evaluate numerically. 

\subsection{Modified Solution Procedure \#1} 
By further modifying the $z_D$-coordinate and performing an exponential substitution, (\ref{eq:mn-d5}) can be simplified significantly and a solution can be found by integration.  Using a scaled dimensionless vertical coordinate 
\begin{equation}\nonumber
 \zeta= a_{kD} z_D \qquad z_D=\frac{\zeta}{a_{kD}} 
\end{equation}
the transformed differential equation representing drawdown in the vadose zone (\ref{eq:mn-d5}) becomes
\begin{equation}
  \label{eq:nondimODE}
  a_{kD}^2 \frac{\mathrm{d}^2
    \bar{\sigma}_D^{\ast}}{\mathrm{d}\zeta^2} - a_{kD}^2
  \frac{\mathrm{d} \bar{\sigma}_D^{\ast}}{\mathrm{d}\zeta} - \left[
    B e^{\lambda_D (1-\zeta/a_{kD})} + C\right] \bar{\sigma}_D^{\ast}=0
\end{equation}
we divide through by $a_{kD}^2$ and performing the substitution
$\bar{\sigma}_D^{\ast}(\zeta)=e^{u(\zeta)}$ (e.g., see \cite[p.\ 27]{bender1978advanced} and \cite[\S 60]{zwillinger1998handbook}).  This exponential change of variables implies
\begin{equation}\nonumber
\frac{\mathrm{d}\bar{\sigma}_D^{\ast}}{\mathrm{d}\zeta} =
\frac{\mathrm{d}u}{\mathrm{d}\zeta}e^u
\end{equation} 
and therefore
\begin{equation}\nonumber
\frac{\mathrm{d}^2\bar{\sigma}_D^{\ast}}{\mathrm{d}\zeta^2} =
\left[ \frac{\mathrm{d}^2u}{\mathrm{d}\zeta^2} +
\left(\frac{\mathrm{d}u}{\mathrm{d}\zeta}\right)^2 \right]e^u
\end{equation}
which changes \eqref{eq:nondimODE} to
\begin{equation}\nonumber
 \left[ \frac{\mathrm{d}^2u}{\mathrm{d}\zeta^2} +
  \left(\frac{\mathrm{d}u}{\mathrm{d}\zeta}\right)^2 \right]e^u -  \left[
  \frac{\mathrm{d}u}{\mathrm{d}\zeta}e^u \right]- \left[ B e^{\lambda_D (1-\zeta/a_{kD})} + C\right] \frac{e^{u}}{a_{kD}^2} =0.
\end{equation}
Multiplying this by $e^{-u}$ and simplifying results in
\begin{equation}
  \label{eq:expsubODE}
   \frac{\mathrm{d} v}{\mathrm{d}\zeta} + v^2 - v = \left[ \frac{B}{a_{kD}^2} e^{\lambda_D (1-\zeta/a_{kD})} + \frac{C}{a_{kD}^2}\right],
\end{equation}
which is known as the Riccati equation, where $v=u'$.  This would only be solvable if a solution could first be guessed, then reducing \eqref{eq:expsubODE} to a Bernoulli equation, which are always solvable.

A simple solution, that is not a Bessel function, might be guessed but I tried a few things and made no progress.

\subsection{Modified Solution Procedure \#2} 
By further modifying the $z_D$-coordinate and performing a different exponential substitution, (\ref{eq:mn-d5}) can be changed to the same form as the time-independent Schr\"{o}dinger equation.  Using a slightly different scaled dimensionless vertical coordinate 
\begin{equation}\nonumber
 \xi= \frac{a_{kD}}{2} z_D \qquad z_D=\frac{2\xi}{a_{kD}} 
\end{equation}
the transformed differential equation representing drawdown in the vadose zone (\ref{eq:mn-d5}) becomes
\begin{equation}
  \label{eq:nondimODE2}
  \frac{a_{kD}^2}{4} \frac{\mathrm{d}^2
    \bar{\sigma}_D^{\ast}}{\mathrm{d}\xi^2} - \frac{a_{kD}^2}{2}
  \frac{\mathrm{d} \bar{\sigma}_D^{\ast}}{\mathrm{d}\xi} - \left[
    B e^{\lambda_D (1-2\xi/a_{kD})} + C\right] \bar{\sigma}_D^{\ast}=0
\end{equation}
we multiply through by $4/a_{kD}^2$ and performing the substitution $\bar{\sigma}^{\ast}_{D} = H(\xi) e^\xi$.  This exponential change of variables (e.g., see \cite{kuhlman2008quasilinear}) implies
\begin{equation}
  \nonumber
  \frac{\mathrm{d} \bar{\sigma}^{\ast}_{D}}{\mathrm{d} \xi} = \frac{\mathrm{d}H}{\mathrm{d}\xi} e^\xi + H e^\xi  
\end{equation}
and therefore
\begin{equation}
  \nonumber
  \frac{\mathrm{d}^2 \bar{\sigma}^{\ast}_{D}}{\mathrm{d} \xi^2} = \frac{\mathrm{d}^2H}{\mathrm{d}\xi^2} e^\xi + 2\frac{\mathrm{d}H}{\mathrm{d}\xi} e^\xi + He^\xi  
\end{equation}
which changes \eqref{eq:nondimODE2} to the form
\begin{equation}
  \label{eq:schrodinger}
  \frac{d^2 H}{d \xi^2} - \left[B' e^{\lambda_D (1-2\xi/{kD})} + C' \right]H = 0
\end{equation}
where $B' = 4B/a_{kD}^2$ and $C'=4C/a_{kD}^2 + 1$.  \eqref{eq:schrodinger} is in the form of the time-independent Schr\"{o}dinger equation.  The $\xi$-dependent coefficient in brackets is referred to as the potential function in the quantum-mechanics literature.  Closed-form solutions have been derived for different potential functions ($1/\xi$, $1/\xi^2$, $e^{-\xi}$, etc.).  

A common solution technique is to transform \eqref{eq:schrodinger} into Bessel's equation and express the solution in terms of Bessel functions (\textit{no help here}).

\section{Saturated Zone Unconfined Solution}
Following \cite{mishra10}, we decompose the saturated zone solution into two solutions: 1) the Hantush solution $s_H$ for a confined aquifer and a partially penetrating well; 2) an unconfined solution $s_U$ with no well but continuity conditions with the vadose zone solution just derived.  The solution for $s_H$ is given elsewhere, either in terms of a finite cosine transform, or a multi-layer fully penetrating solution.  

The governing ordinary differential equation in Laplace-Hankel
transform space for the saturated unconfined $\bar{s}^{\ast}_{UD}$ is
\begin{equation}
  \label{eq:aquifer}
  \frac{\mathrm{d}^2 \bar{s}_{UD}^{\ast}}{\mathrm{d} z_D^2} - \left( \frac{p + a^2}{\kappa} \right)\bar{s}_U^{\ast} = 0 \qquad 0 \le z_D \le 1
\end{equation}
with the no-flow boundary condition at the base of the aquifer
\begin{equation}\nonumber
 \frac{\mathrm{d} \bar{s}_{UD}^{\ast}}{\mathrm{d} z_D} = 0 \qquad z_D=0 
\end{equation}
and the following continuity equations at the water table
\begin{equation}
  \label{eq:headCont}
  \bar{s}_{UD}^{\ast} + \bar{s}_{HD}^{\ast} = \bar{\sigma}_D^{\ast} \qquad z_D=1
\end{equation}
\begin{equation}
  \label{eq:fluxCont}
  \frac{\mathrm{d} \bar{s}_{UD}^{\ast}}{\mathrm{d} z_D}  = \frac{\mathrm{d}
    \bar{\sigma}_D^{\ast}}{\mathrm{d} z_D} \qquad \rightarrow \qquad \frac{\mathrm{d} \bar{s}_{UD}^{\ast}}{\mathrm{d} z_D}  = \frac{\mathrm{d}u}{\mathrm{d} \zeta} e^u \qquad z_D=1
  \qquad 
\end{equation}
note the vertical flux of $\bar{s}_{HD}^{\ast}$ is defined as zero at
the top and bottom of the aquifer, so it is not included in the
continuity condition \eqref{eq:fluxCont}.

The general solution to \eqref{eq:aquifer} is 
\begin{equation}
  \label{eq:su}
  \bar{s}_{UD}^{\ast} = A_1 \cosh(\eta z_D) + A_2 \sinh(\eta z_D)
\end{equation}
where $A_i$ are constants to determine and $\eta^2 = \frac{p + a^2}{\kappa}$.  The no-flow boundary condition at the bottom of the aquifer forces $A_2\equiv 0$.  

\section{Formulation of Finite Difference Solution}
The ordinary differential equation in the vadose zone can be solved
via finite differences in space, while still within Laplace-Hankel
space.  This is done mostly as a check of the other derivation, and
partly because of inspiration derived from a pithy remark by a co-worker.

Substituting finite difference approximations for the $z_D$
derivatives in \eqref{eq:mn-d5} gives
\begin{equation}
  \nonumber
  \frac{1}{h^2} \left( \sigma_{j+1} - 2 \sigma_j + \sigma_{j+1}
  \right) - \frac{a_{kD}}{h} \left( \sigma_{j+1} - \sigma_j\right) -
  \left( B e^{-\lambda_D jh} + C \right) \sigma_j
  = 0
\end{equation}
where the bar, star, and $D$ decorators on $\sigma$ are left out for simplicity, $h$ is
the dimensionless inter-node spacing (assumed constant), and
the subscript indicates the current node $j$ and the node one above
$j+1$ and below $j-1$.  The node index run $0 \le j \le N-1$, where
 there are $N$ nodes and $h = L_D / (N - 1)$.

In the following $\xi_j = B e^{-\lambda_D jh} + C$.
Explicitly writing out the finite-difference matrix for a 3-node
problem with node 1 at $z_D=1$ and node 3 at $z_D=1+L/b$, without
explicitly handling the boundary conditions, is
\begin{equation}
  \label{eq:fd-01}
  \left[ \begin{matrix}
    \frac{1}{h^2} & \left(\frac{a_{kD}}{h} - \frac{2}{h^2} - \xi_0\right) &
    \frac{1}{h^2} - \frac{a_{kD}}{h} & 0 & 0 \\ 
    0 & \frac{1}{h^2} & \left(\frac{a_{kD}}{h} - \frac{2}{h^2} - \xi_1 \right)&
    \frac{1}{h^2} - \frac{a_{kD}}{h} & 0  \\ 
    0 & 0 & \frac{1}{h^2} & \left(\frac{a_{kD}}{h} - \frac{2}{h^2}  -
      \xi_2 \right) &
    \frac{1}{h^2} - \frac{a_{kD}}{h} \\ 
  \end{matrix}\right] 
\left[\begin{matrix}
\sigma_{-1} \\ \sigma_0 \\ \sigma_1 \\ \sigma_2 \\ \sigma_3
\end{matrix}\right]
=
\left[\begin{matrix}
0 \\ 0\\ 0 \\ 0 \\ 0
\end{matrix}\right]
\end{equation}
where $\sigma_{-1}$ and $\sigma_3$ are ``ghost'' nodes
which will be used to represent boundary conditions, but do not
physically exist; \eqref{eq:fd-01} is in the canonical form $\mathbf{Ax}=\mathbf{b}$.  Because
$\partial \sigma/\partial z =0$ at $z = b+L$, we can set $\sigma_2=
\sigma_3$, which means adding the two rightmost columns of
$\mathbf{A}$ and eliminating $\sigma_3$.

The node at the bottom of the vadose zone ($\sigma_0$) appears in two
boundary conditions, the head and flux continuity conditions with the
aquifer.  The flux continuity condition can be written in
finite-difference form as
\begin{equation}
 \nonumber
  \lim_{h \rightarrow 0} \frac{\sigma_0 - \sigma_{-1}}{h} = A_1 \eta \sinh(\eta)
\end{equation}
which can be used to eliminate $\sigma_{-1}$ from $\mathbf{A}$.
Substituting $\sigma_{-1} = \sigma_0 - hA_1 \eta \sinh(\eta)$ into
\eqref{eq:fd-01} gives the following equation for the first row of
$\mathbf{A}$
\begin{equation}
  \nonumber
  \frac{1}{h^2} \left[ \sigma_0 - hA_1 \eta \sinh(\eta)\right] +
  \left(\frac{a_{kD}}{h} - \frac{2}{h^2}  - \xi_0\right) \sigma_0
  + \left(\frac{1}{h^2} - \frac{a_{kD}}{h}\right) \sigma_1 = 0
\end{equation}
Further using head continuity equation $\sigma_0 = A_1 \cosh(\eta) +
s_{H}(z_D=1)$ to eliminate $\sigma_0$ from this first row leads to
\begin{equation}
 \nonumber
    \frac{A_1}{h}  \eta \sinh(\eta) +
  \left( \frac{a_{kD}}{h} - \frac{1}{h^2} + - \xi_0\right) \left[
    A_1 \cosh(\eta) + s_{H}(1)\right] + \left(\frac{1}{h^2} -
    \frac{a_{kD}}{h}\right) \sigma_1 = 0.
\end{equation}
Performing the substitution for $\sigma_0$ into the second row of
$\mathbf{A}$, and regrouping to solve for $A_1$ in place of the
eliminated $\sigma_0$ leads to
 \begin{equation}
  \label{eq:fd-02}
  \left[ \begin{matrix}
     \cosh(\eta)\theta_1 - \frac{\eta}{h} \sinh(\eta) &
    \frac{1}{h^2} - \frac{a_{kD}}{h} & 0 \\ 
     \frac{1}{h^2}\cosh(\eta) & \theta_2 &
    \frac{1}{h^2} - \frac{a_{kD}}{h}   \\ 
     0 & \frac{1}{h^2} & \theta_2 +
    \frac{1}{h^2} - \frac{a_{kD}}{h} \\ 
  \end{matrix}\right] 
\left[\begin{matrix}
A_1 \\ \sigma_1 \\ \sigma_2 
\end{matrix}\right]
=
\left[\begin{matrix}
-\theta_1 s_H(1) \\ 
-\frac{1}{h^2}s_H(1) \\ 0 
\end{matrix}\right]
\end{equation}
where $\theta_1 = \frac{a_{kD}}{h} - \frac{1}{h^2} - \xi_j$ and $\theta_2 =
\frac{a_{kD}}{h} - \frac{2}{h^2} - \xi_j $.  This 
$3\times 3$ case illustrates all the entries in a general $n\times
n$ problem; including boundary nodes and an interior node.  

The upper--left coefficient $\cosh(\eta)\theta_1 - \frac{\eta}{h}\sinh(\eta)$ in
\eqref{eq:fd-02} may be problematic due to cancellation between large
quantities, so it can be re-written in terms of exponentials as 
\begin{equation}
 \nonumber
 \frac{1}{2} \left[ e^{\eta} \left( \theta_1 - \frac{\eta}{h}\right) + e^{-\eta}
   \left( \theta_1 + \frac{\eta}{h}\right) \right].
\end{equation}

The Thomas algorithm for tri-diagonal matrices can be used to
numerically solve for $A_1$ and $\sigma_j$, where $j=1,2, \dots, n$
(not including $\sigma_0$).  Once $A_1$ is found numerically, the
solution in the saturated zone is simply 
\begin{equation}
 \nonumber
 \bar{\sigma}^{\ast}_{D}(z_D) = \bar{\sigma}^{\ast}_{HD}(z_D) + A_1
 \cosh(\eta z_D).
\end{equation}
This involves computing the Thomas algorithm in Laplace-Hankel space,
for each combination of $(a,p)$.  The Thomas algorithm can be
vectorized to compute many solutions simultaneously.

For small problems ($n \le 5$) the finite difference matrix above can
be solved algebraically using Cramer's rule, or using equivalently
using Mathematica.  For example, the matrix \eqref{eq:fd-02} for three nodes $(h=L_D/2)$ can be solved for $A_1$ analytically as
\begin{align}
  \label{eq:AsolnMathematica}
  A_1&=\left\lbrace-\frac{4 \left(4-2 a_k L_D\right) \left[4+\left(C+B e^{-L_D \lambda _D}\right) L_D^2\right]}{L_D^6}+\left(B+C+\frac{4-2 a_k L_D}{L_D^2}\right) \right.  \times \notag\\
      & \left.\left.\left[\frac{8 \left(-2+a_k L_D\right)}{L_D^4}+\left(-C-B e^{-L_D \lambda _D}-\frac{4}{L_D^2}\right) \left(-C-B e^{-\frac{1}{2} L_D \lambda _D}+\frac{2 \left(-4+a_k L_D\right)}{L_D^2}\right)\right]\right\rbrace s_H \right/\notag\\
  &\left\lbrace\frac{4 \cosh(\eta) \left(4-2 a_k L_D\right) \left(4+\left(C+B e^{-L_D \lambda _D}\right) L_D^2\right)}{L_D^6}+\right. \\
    &\frac{-4 \cosh(\eta)+L_D \left(-2 \eta  \sinh(\eta)+2 \cosh(\eta) a_k-(B+C) \cosh(\eta) L_D\right)}{L_D^2} \notag\\
    &\left.\left[\frac{8 \left(-2+a_k L_D\right)}{L_D^4}+\left(-C-B e^{-L_D \lambda _D}-\frac{4}{L_D^2}\right) \left(-C-B e^{-\frac{1}{2} L_D \lambda _D}+\frac{2 \left(-4+a_k L_D\right)}{L_D^2}\right)\right]\right\rbrace \notag
\end{align}
which would only be accuracte if the vadose zone was small, so that $L_D/2 \ll 1$. 

\bibliographystyle{alpha}
\bibliography{mn}

\end{document}
